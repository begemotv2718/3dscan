\documentclass{article}

\begin{document}
\title A project of 3d scanner
\maketitle
\section{Geometry}
\subsection{Linear perspective projection}
Assuming that the camera is placed at $(x,y,z)=(0,0,0)$ and optical axis of the camera is $X$ axis  we get the following equations for the coordinates of the projection of the point $(x,y,z)$
\begin{equation}
\begin{array}{l}
 y_{im}=\frac{y f}{x} \\
 z_{im} = \frac{ z f}{x}
\end{array}
\end{equation}
\subsection{Reconstruction of the point on the given plane by projection}
Let the plane has equation $\vec{n}\cdot\vec{r}=d$. All the points that projects into point $(y_{im}, z_{im})$ of the image are on the line that can be represented by parametrical equation $\vec{r}=(1, \frac{y_{im}}{f}, \frac{z_{im}}{f})t=\vec{m}t$. Thus the coordinate of the intersection is 
\begin{equation}
\vec{r} = \frac{\vec{m} d}{\vec{n}\cdot\vec{m}}
\end{equation}
\subsection{Reconstruction of the plane from intersection lines with two given planes}
We first consider reconstruction from the traces left by the
intersection of the plane with two mutually perpendicular planes. The
line of intersection of the perpendicular planes are assumed to be
parallel to $z$-axis and to intersect with the camera optical axis at
the distance $d$. Let the angle between the first plane and camera
axis be $\alpha$, thus the angle between the second plane and the
camera axis is $\pi/2-\alpha$. 

Consider the point at distance $\rho$ from the line of intersection, lying on the first plane. The coordinates of this point will be 
$(d-\rho \cos \alpha, -\rho \sin\alpha, z)$. The projection of such a point 

\appendix
\section{Rational function}
Consider the following function
\begin{equation}
\label{eq:rat}
y = \frac{a x + b}{c x +d }.
\end{equation} 
We can formally represent this function as a matrix
\begin{equation}
A = \left(
  \begin{array}{ll}
    a  & b \\
    c  & d 
  \end{array}
\right).
\end{equation}
Now lets assume that we make substitution in~(\ref{eq:rat})
\[ x = \frac{ a_1 t + b_1 } { c_1 t +d_1} \] the matrix for this substitution being
\begin{equation}
A_1 = \left(
  \begin{array}{ll}
    a_1  & b_1 \\
    c_1  & d_1 
  \end{array}
\right).
\end{equation}
Then the resulting matrix representing $y(t)$ will be $A A_1$.

It is obvious that  $x(y)$ for  Eq.~(\ref{eq:rat}) is represented by the matrix $A^{-1}$. It is also obvious that the matrix is defined up to a constant multiplier, thus, for uniquiness one may consider matrices with unit determinant ( the case of zero determinant is degenerate). Thus, rational functions are related to group $SL(2)$.

\end{document}